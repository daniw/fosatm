% coding:utf-8

%----------------------------------------
%FOSAMATH, a LaTeX-Code for a mathematical summary for basic analysis
%Copyright (C) 2013, Daniel Winz, Ervin Mazlagic, Adrian Imboden, Philipp Langer

%This program is free software; you can redistribute it and/or
%modify it under the terms of the GNU General Public License
%as published by the Free Software Foundation; either version 2
%of the License, or (at your option) any later version.

%This program is distributed in the hope that it will be useful,
%but WITHOUT ANY WARRANTY; without even the implied warranty of
%MERCHANTABILITY or FITNESS FOR A PARTICULAR PURPOSE.  See the
%GNU General Public License for more details.
%----------------------------------------

% coding:utf-8
\section{Zug und Druck}
Voraussetzungen: 
\begin{itemize}
  \item Querschnittsfläche senkrecht zur Stabachse
  \item Normalkraft fällt mit den Flächenmittelpunkt zusammen
\end{itemize}
Spannungsverteilung im (glatten) Zugstab ist über den Querschnitt konstant
\[ \sigma = \frac{F}{A} \]
Verlängerung des Stabes
\[ \Delta L = \frac{F \cdot \ell}{E \cdot A} \]
Dimensionierung des erforderlichen Querschnitts
\[ A_{erf} \geq \frac{F}{\sigma_{zul}} \]
Bei gekerbten Stäben ist die Spannungserhöhung zu berücksichtigen

\subsection{Vorzeichen der Spannung}
Zug: positive Spannung\\
Druck: negative Apannung\\\\
Gleichungen gelten für Zug und Druck\\\\
Schlanken Druckstäbe können knicken, bevor sie die Festigkeit auf Druck überschreiten. (Knickung)

\subsection{Beliebiger Schnitt}
Normalspannung unter dem Winkel $\alpha$
\[ \sigma = \frac{\sigma_0}{2} \cdot (1 + \cos 2 \alpha) \]
Schubspannung unter dem Winkel $\alpha$
\[ \tau = \frac{\sigma_0}{2} \cdot \sin 2 \alpha \]
Maximalwerte
\[ \begin{array}{llll}
\alpha = 0 & \rightarrow & \sigma_{max} = \sigma_0 & \tau = 0\\
\alpha = 45^{\circ} & \rightarrow & \tau_{max} = \frac{\sigma_0}{2} & \sigma = \frac{\sigma_0}{2}
\end{array} \]
Zähe Werkstoffe, z.B. c) Baustahl versagen durch Abgleiten und Anriss unter $45^{\circ}$\\
Bei spröden Werkstoffen verläuft der Bruch bei Zugbelastung senkrecht zur Spannungsrichtung wie zum Beispiel bei Kreide oder Grauguss 

\subsection{Wärmedehnung}
Thermische Verlängerung
\[ \Delta L_{th} = \alpha \cdot L_0 \cdot \Delta T \]
Thermische Dehnung
\[ \varepsilon = \alpha \cdot \Delta T \]
Überlagerung mit der elastischen Dehnung
\[ \varepsilon_{ges} = \varepsilon_{el} + \varepsilon_{th} \]

\subsection{Wärmespannungen}
Thermische Verlängerung
\[ \Delta L_{th} = \alpha \cdot L_0 \cdot \Delta T \]
Wird eliminiert durch elastische Stauchung
\[ \Delta L_{el} = \frac{F \cdot L_0}{E \cdot A} \]
\[ \Delta L_{th} = \Delta L_{el} \rightarrow \alpha \cdot L_0 \cdot \Delta T = \frac{F \cdot L_0}{E \cdot A} \]
Spannung
\[ \sigma = \frac{F}{A} = \alpha \cdot \Delta T \cdot E \]
F ist eine Druckkraft $\rightarrow$ Spannung negativ
\[ \sigma = - \sigma \cdot \Delta T \cdot E \]

\subsection{Flächenpressung}
Die Flächenpressung bei zwei Oberflächen kann sehr komplex sein. \\
Für die Dimensionierung wird mit einer mittleren Flächenpressung gerechnet. 
\[ p = \frac{F}{A} \]

\subsection{Lochleibung}
Die Flächenpressung eines Bolzens in einer Bohrung ist rechnerisch schwierig zu erfassen. 
Für Berechnungen  wird die Kraft auf die projizierte Fläche angewendet. Die sich ergebende Grösse wird Lochleibungsdruck genannt. 
\[ p_l = \frac{F}{d \cdot \ell} \]
