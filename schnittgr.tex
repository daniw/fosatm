% coding:utf-8

%----------------------------------------
%FOSAMATH, a LaTeX-Code for a mathematical summary for basic analysis
%Copyright (C) 2013, Daniel Winz, Ervin Mazlagic, Adrian Imboden, Philipp Langer

%This program is free software; you can redistribute it and/or
%modify it under the terms of the GNU General Public License
%as published by the Free Software Foundation; either version 2
%of the License, or (at your option) any later version.

%This program is distributed in the hope that it will be useful,
%but WITHOUT ANY WARRANTY; without even the implied warranty of
%MERCHANTABILITY or FITNESS FOR A PARTICULAR PURPOSE.  See the
%GNU General Public License for more details.
%----------------------------------------

% coding:utf-8
\section{Schnittgrössen}

\subsection{Koordinatensystem}
Für Balken ist üblich, dass die x-Achse in Längsrichtung des Balkens (nach rechts) liegt und die y-Achse nach unten zeigt. Das Moment zeigt in Gegenuhrzeigerrichtung. 

\subsection{Schnittufer}
\begin{tabular}{lll}
\rowcolor{white} \textbf{Schnittgrösse} & \textbf{Vektor} & \textbf{Achsrichtung}\\
\rowcolor{lgray}
\textcolor{green} {positiv} & 
\textcolor{green} {positiv} & 
\textcolor{green} {positiv} \\
\rowcolor{white}
\textcolor{green} {positiv} & 
\textcolor{red}   {negativ} & 
\textcolor{red}   {negativ} \\
\rowcolor{lgray}
\textcolor{red}   {negativ} & 
\textcolor{green} {positiv} & 
\textcolor{red}   {negativ} \\
\rowcolor{white}
\textcolor{red}   {negativ} & 
\textcolor{red}   {negativ} & 
\textcolor{green} {positiv} 
\end{tabular}

\subsection{Schnittgrössen am Balken}
\[ \sum F_x: N - C_x = 0 \]
\[ \sum F_z: Q - C_z = 0 \]
\[ \sum M_B: M_b + M_C - C_z \cdot b = 0 \]

\subsection{Vorgehen bei der Berechnung von Schnittgrössen}
\begin{enumerate}
  \item Lagerreaktionen berechnen
  \item Koordinatensystem einzeichnen
  \item äussere Kräfte auf Bsuteil einzeichnen
  \item Schnittkräfte einzeichnen
  \item Gleichgewichtsbedingungen aufstellen
  \item Schnittkräfte berechnen
\end{enumerate}