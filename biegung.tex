% coding:utf-8

%----------------------------------------
%FOSAMATH, a LaTeX-Code for a mathematical summary for basic analysis
%Copyright (C) 2013, Daniel Winz, Ervin Mazlagic, Adrian Imboden, Philipp Langer

%This program is free software; you can redistribute it and/or
%modify it under the terms of the GNU General Public License
%as published by the Free Software Foundation; either version 2
%of the License, or (at your option) any later version.

%This program is distributed in the hope that it will be useful,
%but WITHOUT ANY WARRANTY; without even the implied warranty of
%MERCHANTABILITY or FITNESS FOR A PARTICULAR PURPOSE.  See the
%GNU General Public License for more details.
%----------------------------------------

% coding:utf-8
\section{Biegung}
Reine Biegung: Keine Querkraft\\
Querkraftbiegung: Belstung zusammengesetzt aus Biegung und Querkraft\\\\
Innere Kraft (jeweils obere und untere Querschnittshälfte)
\[ F_i = \frac{1}{2} \cdot \sigma_{max} \cdot \frac{h}{2} \cdot b \]
Abstand der inneren Kräfte (dreieckförmige Spannungsverteilung)
\[ e = 2 \cdot \frac{2}{3} \cdot \frac{h}{2} = \frac{2}{3} \cdot h \]
Biegemoment
\[ M_b = F_i \cdot e = \frac{1}{2} \cdot \sigma_{max} \cdot \frac{h}{2} \cdot b \cdot \frac{2}{3} \cdot h \]
Maximale Biegespannung für den Rechteck-Querschnitt
\[ \sigma_{max} = \frac{M_b}{\left(\frac{b \cdot h^2}{6}\right)} = \frac{M_b}{W_y} \]
Widerstandsmoment eines rechteckigen Querschnitts $(W_y)$
\[ W_y = \left(\frac{b \cdot h^2}{6}\right) \]
Die neutrale Achse $(\sigma = 0)$ liegt im Flächenmittelpunkt. Beim Rechteckquerschnitt liegt dieser in der Mitte. 

\subsection{Biegespannung}
\sigma_{max} = \frac{M_b}{W_y}
Biegespannung in Funktion der Höhe ($z$)
\[ \sigma = \frac{M_b}{I_y} \cdot z \]
Zusammenhang zwischen Widerstandsmoment $(W_y)$ und Flächenmoment $(I_y)$
\[ W_y = \frac{I_y}{z_{max}} \]
$z_{max}$ ist der maximale Abstand von der neutralen Achse. 